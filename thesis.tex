\documentclass[11pt]{extbook}
\renewcommand{\baselinestretch}{1.04}

\usepackage{verbatim} % For comments

\usepackage[dutch,british]{babel}

% for booklet
%\usepackage[paperheight=240mm,paperwidth=170mm,top=20mm,bottom=20mm,right=20mm,left=20mm]{geometry}
%\usepackage[height=250mm,width=180mm,cam,noinfo,pdftex,center]{crop}

%%%% Things I did for reading version%%%%%%%%%%%%
\usepackage[paperheight=297mm,paperwidth=210mm,top=25mm,bottom=25mm,right=25mm,left=25mm]{geometry}
%\usepackage{pdfpages}
%%%%%%%%%%%%%%%%%%


\usepackage{nameref}
\usepackage[hypertexnames=false,hidelinks,bookmarks]{hyperref}

\usepackage{etoolbox}
% Fix multiple links for references issue.
\makeatletter
\pretocmd{\NAT@citex}{%
  \let\NAT@hyper@\NAT@hyper@citex
  \def\NAT@postnote{#2}%
  \setcounter{NAT@total@cites}{0}%
  \setcounter{NAT@count@cites}{0}%
  \forcsvlist{\stepcounter{NAT@total@cites}\@gobble}{#3}}{}{}
\newcounter{NAT@total@cites}
\newcounter{NAT@count@cites}
\def\NAT@postnote{}

% include postnote and \citet closing bracket in hyperlink
\def\NAT@hyper@citex#1{%
  \stepcounter{NAT@count@cites}%
  \hyper@natlinkstart{\@citeb\@extra@b@citeb}#1%
  \ifnumequal{\value{NAT@count@cites}}{\value{NAT@total@cites}}
    {\ifNAT@swa\else\if*\NAT@postnote*\else%
     \NAT@cmt\NAT@postnote\global\def\NAT@postnote{}\fi\fi}{}%
  \ifNAT@swa\else\if\relax\NAT@date\relax
  \else\NAT@@close\global\let\NAT@nm\@empty\fi\fi% avoid compact citations
  \hyper@natlinkend}
\renewcommand\hyper@natlinkbreak[2]{#1}

% avoid extraneous postnotes, closing brackets
\patchcmd{\NAT@citex}
  {\ifNAT@swa\else\if*#2*\else\NAT@cmt#2\fi
   \if\relax\NAT@date\relax\else\NAT@@close\fi\fi}{}{}{}
\patchcmd{\NAT@citex}
  {\if\relax\NAT@date\relax\NAT@def@citea\else\NAT@def@citea@close\fi}
  {\if\relax\NAT@date\relax\NAT@def@citea\else\NAT@def@citea@space\fi}{}{}
\makeatother

\usepackage{graphicx}
\usepackage{longtable}
\usepackage{tabularx}
\usepackage{floatrow}
\DeclareFloatSeparators{captionmargin}{\hspace{2.5mm}} 
\usepackage{adjustbox}
\usepackage{float}
\floatstyle{plaintop}
\restylefloat{table}
\usepackage{pbox}
\usepackage{wrapfig}
\setlength{\textfloatsep}{16pt}
\setlength{\intextsep}{16pt}
\renewcommand{\bottomfraction}{0.7}
\usepackage[ansinew]{inputenc}

\usepackage{multirow}
\usepackage{booktabs}
%\usepackage{placeins}

% Symbol packages
\usepackage[fleqn]{amsmath}
\usepackage{amssymb}
\usepackage{upgreek}
\usepackage{bm}
\usepackage{mathtools}
\usepackage{nccmath}
\usepackage{array}   % for \newcolumntype macro
\newcolumntype{L}{>{$}l<{$}} % math-mode version of "l" column type
\newcolumntype{C}{>{$}c<{$}} % math-mode version of "l" column type

% Set fonts.
\usepackage{ebgaramond}
\usepackage{uarial}
\usepackage[T1]{fontenc}
\newcommand{\fakebf}{\fontfamily{mdugm}\fontseries{b}\selectfont\small}
\DeclareTextFontCommand{\textbf}{\fakebf}

% landscape package
\usepackage{lscape}

% Set captions and section title fonts.
\usepackage[tableposition=top]{caption}
% Declare sans-serif math version for the captions.
\DeclareMathVersion{captionmath}
\SetSymbolFont{operators}{captionmath}{OT1}{cmbr}{m}{n}
\SetSymbolFont{letters}{captionmath}{OML}{cmbrm}{m}{it}
\SetSymbolFont{symbols}{captionmath}{OMS}{cmbrs}{m}{n}
\SetMathAlphabet\mathbf{captionmath}{OT1}{cmbr}{bx}{n}
\SetMathAlphabet\mathsf{captionmath}{OT1}{cmbr}{m}{n}
\SetMathAlphabet\mathit{captionmath}{OT1}{cmbr}{m}{it}
\SetMathAlphabet\mathtt{captionmath}{OT1}{cmtl}{m}{n}
\DeclareCaptionFont{sansmath}{\mathversion{captionmath}}

%%%%%%%%%%%%%%%%%%%%%%%%%%%%%%%%%%%%%%%%%%%%%%%%%
% Glossaries for list of symbols
\usepackage[symbols,nogroupskip,nonumberlist,automake]{glossaries-extra}
\makeglossaries

\setlength{\glsdescwidth}{0.8\linewidth}
\glsxtrnewsymbol[description={Front-to-back translatory optic flow in a free-flight (s$^{-1}$)}]{tof}{\ensuremath{T_{OF}}}
\glsxtrnewsymbol[description={Set-point of front-to-back translatory optic flow in a free-flight (s$^{-1}$)}]{tofs}{\ensuremath{T^*_{OF}}}
\glsxtrnewsymbol[description={Flight speed in a free-flight condition (m s$^{-1}$)}]{vf}{\ensuremath{V_F}}
\glsxtrnewsymbol[description={Distance from the surface in a free-flight condition (m)}]{D}{\ensuremath{D}}
\glsxtrnewsymbol[description={Time-to-touchdown (s)}]{t}{\ensuremath{t}}
\glsxtrnewsymbol[description={Fit percentage}]{F}{\ensuremath{F}}
\glsxtrnewsymbol[description={Coefficient of determination}]{R2}{\ensuremath{R^2}}
\glsxtrnewsymbol[description={Time-to-contact parameter (s) (inverse of optical expansion rate $r$)}]{tau}{\ensuremath{\tau}}

\glsxtrnewsymbol[description={Time-to-contact-rate (time derivative of time-to-contact $r$)}]{taudot}{\ensuremath{\dot{\tau}}}
\glsxtrnewsymbol[description={Relative rate of expansion or optical expansion rate (s$^{-1}$)}]{r}{\ensuremath{r}}
\glsxtrnewsymbol[description={Optical expansion rate at the start of the entry segment (s$^{-1}$)}]{r0}{\ensuremath{r_0}}
\glsxtrnewsymbol[description={Initial distance from the landing platform at which the entry segment starts (m)}]{y0}{\ensuremath{y_0}}
\glsxtrnewsymbol[description={Average approach acceleration $A$ during an entry segment (m s$^{-2}$)}]{Ameane}{\ensuremath{\overline{A}_{e}}}
\glsxtrnewsymbol[description={Average airspeed $U_A$ during an entry segment (m s$^{-1}$)}]{UAe}{\ensuremath{\overline{U}_{A~e}}}
\glsxtrnewsymbol[description={Change in approach velocity during an entry segment (m s$^{-1}$)}]{deltaV}{\ensuremath{\Delta V}}
\glsxtrnewsymbol[description={Time duration of an entry segment (s)}]{deltat}{\ensuremath{\Delta t}}
\glsxtrnewsymbol[description={Optical expansion acceleration; calculated as time derivative of optical expansion rate (s$^{-2}$)}]{rdot}{\ensuremath{\dot{r}}}
\glsxtrnewsymbol[description={Estimate of optical expansion acceleration in an entry segment (s$^{-2}$)}]{rdote}{\ensuremath{\dot{r}_e}}
\glsxtrnewsymbol[description={Step-change of optical expansion rate required in an entry segment (s$^{-1}$)}]{deltare}{\ensuremath{\Delta r_e}}
\glsxtrnewsymbol[description={Temporal variation of relative rate of expansion in a combined pair of constant-$r$ and entry segments (s$^{-1}$)}]{rct}{\ensuremath{r_c(t)}}
\glsxtrnewsymbol[description={Low-pass filtered temporal variation of relative rate of expansion in a combined pair of constant-$r$ and entry segments (s$^{-1}$)}]{rft}{\ensuremath{r_f(t)}}
\glsxtrnewsymbol[description={Temporal variation of relative rate of expansion in a combined pair of constant-$r$ and entry segments obtained after simulating a transfer function (s$^{-1}$)}]{rst}{\ensuremath{r_s(t)}}
\glsxtrnewsymbol[description={Set-point of relative rate of expansion or optical expansion rate; mean value of optical expansion rate in a constant-$r$ segment  (s$^{-1}$)}]{rs}{\ensuremath{r^*}}
\glsxtrnewsymbol[description={Switch-reversal set-point of optical expansion rate (s$^{-1}$)}]{srs}{\ensuremath{r_0^*}}
\glsxtrnewsymbol[description={Slope of linear relationship between the logarithmic transformations of the set-points of optical expansion rate $r^*$ and the mean distance to the surface $y^*$; a parameter similar to time-to-contact rate $\dot{\tau}$}]{m}{\ensuremath{m}}
\glsxtrnewsymbol[description={Change in set-point of optical expansion rate between two consecutive constant-$r$ segments in a landing approach (s$^{-1}$)}]{deltars}{\ensuremath{\Delta r^*}}
\glsxtrnewsymbol[description={Mean value of distance to the surface in a constant-$r$ segment  (m)}]{ys}{\ensuremath{y^*}}
\glsxtrnewsymbol[description={Time duration during a constant-$r$ segment  (s)}]{ts}{\ensuremath{\Delta t^*}}
\glsxtrnewsymbol[description={Mean value of approach acceleration $A$ in a constant-$r$ segment  (m s$^{-2}$)}]{As}{\ensuremath{A^*}}
\glsxtrnewsymbol[description={Mean value of approach velocity $V$ in a constant-$r$ segment  (m s$^{-1}$)}]{Vs}{\ensuremath{V^*}}
\glsxtrnewsymbol[description={An axis of a landing platform coordinate system; defined in a direction normal to the landing platform (m)}]{y}{\ensuremath{y}}
\glsxtrnewsymbol[description={An axis of a landing platform coordinate system; defined in a plane parallel to the ground (m)}]{x}{\ensuremath{x}}
\glsxtrnewsymbol[description={An axis of a landing platform coordinate system; defined in a vertically up direction (m)}]{z}{\ensuremath{z}}
\glsxtrnewsymbol[description={Approach velocity towards the landing platform (m s$^{-1}$)}]{V}{\ensuremath{V}}
\glsxtrnewsymbol[description={Approach acceleration towards the landing platform (m s$^{-2}$)}]{A}{\ensuremath{A}}
\glsxtrnewsymbol[description={Three-dimensional speed at the start of the landing maneuver (m s$^{-1}$)}]{Ustart}{\ensuremath{U_{start}}}
\glsxtrnewsymbol[description={Average flight speed between two consecutive constant-$r$ segments for a hybrid landing strategy (m s$^{-1}$)}]{UH}{\ensuremath{U_{H}}}
\glsxtrnewsymbol[description={Average flight speed between two consecutive constant-$r$ segments for a constant-$r$ landing strategy (m s$^{-1}$)}]{Ur}{\ensuremath{U_{r}}}
\glsxtrnewsymbol[description={Average flight speed between two consecutive constant-$r$ segments for a constant-$\dot{\tau}$ landing strategy (m s$^{-1}$)}]{Utaudot}{\ensuremath{U_{\dot{\tau}}}}
\glsxtrnewsymbol[description={Displacement normal to the landing surface during a constant-$r$ segment (m)}]{deltay1}{\ensuremath{\Delta y_1} or \ensuremath{\Delta y^*}}
\glsxtrnewsymbol[description={Displacement normal to the landing surface for a set of consecutive constant-$r$ segments (m)}]{deltay2}{\ensuremath{\Delta y_2}}
\glsxtrnewsymbol[description={Threshold factor used in the set-point extraction algorithm}]{f}{\ensuremath{f}}
\glsxtrnewsymbol[description={Space-time array ($x,y,z,t$)}]{vecX}{\ensuremath{\bm{X}}}
\glsxtrnewsymbol[description={Ground-velocity vector ($u,v,w$) or ($u_G,v_G,w_G$) in landing platform coordinate system (m s$^{-1}$)}]{vecU}{\ensuremath{\bm{U}} or \ensuremath{\bm{U_G}}}
\glsxtrnewsymbol[description={Wind-velocity vector ($u_W,0,0$) in landing platform coordinate system (m s$^{-1}$)}]{vecUW}{\ensuremath{\bm{U_W}}}
\glsxtrnewsymbol[description={Air-velocity vector ($u_A,v_A,w_A$) in landing platform coordinate system (m s$^{-1}$)}]{vecUA}{\ensuremath{\bm{U_A}}}
\glsxtrnewsymbol[description={Acceleration vector ($a_x,a_y,a_z$) (m s$^{-2}$)}]{vecA}{\ensuremath{\bm{A}}}
\glsxtrnewsymbol[description={Parameters of a gamma distribution}]{ab}{\ensuremath{a,~b}}
\glsxtrnewsymbol[description={Probability of a bumblebee exhibiting low approach velocity phase}]{PlowV}{\ensuremath{P_{\textrm{low }V}}}
%%%%%%%%%%%%%%%%%%%%%%%%%%%%%%%%%%%%%%%%%%%%%%%%%

\captionsetup{figurename=Figure,font={scriptsize,sf,sansmath},labelfont=bf}

% Make sure equation labels are always normal size, even if the equations themselves are smaller.
\makeatletter
\def\maketag@@@#1{\hbox{\m@th\normalfont\normalsize#1}}
\makeatother

% Format headings.
\usepackage[noindentafter]{titlesec}
% Fix bug
\makeatletter
\patchcmd{\ttlh@hang}{\parindent\z@}{\parindent\z@\leavevmode}{}{}
\patchcmd{\ttlh@hang}{\noindent}{}{}{}
\makeatother
\titleformat{\chapter}{\vspace{-50pt}\Huge}{\thechapter}{4mm}{\Huge\vspace{-30pt}}

\titleformat{\section}{\normalfont\sffamily\Large\bfseries}{\thesection}{4mm}{}
\titlespacing{\section}{0pt}{11pt plus 6pt minus 2pt}{6pt plus 0pt minus 0pt}
\titleformat{\subsection}{\normalfont\sffamily\large\bfseries}{\thesubsection}{3mm}{}
\titlespacing{\section}{0pt}{8pt plus 4pt minus 2pt}{6pt plus 0pt minus 0pt}
\titleformat{\subsubsection}{\normalfont\sffamily\normalsize\bfseries}{\thesubsubsection}{3mm}{}
\titlespacing{\section}{0pt}{8pt plus 4pt minus 2pt}{6pt plus 0pt minus 0pt}

% Define \thumb command for use in the header
\usepackage{tikz}
\usetikzlibrary{calc}
\usepackage{xcolor}

\newcommand{\thumb}[2]{%
\begin{tikzpicture}[remember picture,overlay]
\fill[fill=lightgray]
($(current page.north east) + (-10mm,-5mm) + (0,#2mm)$) 
-- ($(current page.north east) + (5mm,-5mm) + (0,#2mm)$)
-- ($(current page.north east) + (5mm,5mm) + (0,#2mm)$)
-- ($(current page.north east) + (-10mm, 5mm) + (0,#2mm)$)
-- cycle;
\node(test) at ($(current page.north east) + (-5mm,#2mm)$){\Huge\textcolor{white}{#1}};
\end{tikzpicture}}

% Fish flickbook
\newcommand{\fish}{%
\begin{tikzpicture}[remember picture,overlay]
\node(test) at ($(current page.north west) + (10mm,-120mm)$){\includegraphics[width=20mm]{figures/flickbook/frame\thepage}};
\end{tikzpicture}}
%\renewcommand{\fish}{\textit{fish \thepage}}

% Title page
\newcommand{\titleimage}[1]{%
\begin{tikzpicture}[remember picture,overlay]
\node(test) at ($(current page.center)$){\includegraphics[width=180mm]{#1}};
\end{tikzpicture}}

% Customise headers and footers
\usepackage{fancyhdr}

\fancypagestyle{fishempty}{%
\fancyhf{}
\fancyhead[LE]{\fish}
\renewcommand{\headrulewidth}{0pt}
}

\pagestyle{fancy}

\fancyhf{}
\fancyfoot[LE,RO]{\thepage}

%%%% Things I did for reading version%%%%%%%%%% remove them for final version
%\fancyfootoffset{2.5cm}
%%%%%%%%%%%%%%%%%%%%%%%%%%%%%%%%%%%%%%%%%%%%%%%%%%

\newcommand{\chapterthumb}{}
%\newcommand{\chapterthumb}{placeholder for page thumbs}
\fancyhead[CO]{\chapterthumb}
%\fancyhead[LE]{\fish}
\renewcommand{\headrulewidth}{0pt}

\makeatletter
\let\ps@plain\ps@fancy
\makeatother

\makeatletter
\newcommand{\cleardoublepageeven}{\clearpage\if@twoside\ifodd\c@page \hbox{}\newpage\if@twocolumn\hbox{}\newpage\fi\fi\fi}
\makeatother

% Remove chapter heading for chapters with a (manually defined) title page - these will be added manually later.
\makeatletter
\newcommand{\unchapter}[1]{%
  \begingroup
  %\let\ps@plain\ps@empty
  %\let\ps@fancy\ps@empty
  \renewcommand{\cleardoublepage}{\cleardoublepageeven}
  \let\@makechapterhead\@gobble % make \@makechapterhead do nothing
  \chapter{#1}
  \endgroup
}
\makeatother

% Make fake chapter heading for the pageref in the TOC, so we can insert empty pages for the title page.
\makeatletter
\newcommand{\chapterpagemark}[1]{%
  \begingroup
  %\let\ps@plain\ps@empty
  %\let\ps@fancy\ps@empty
  % Make sure our chapters start at the even page by redefining \cleardoublepage, which is used in the macro @makechapterhead.
  \renewcommand{\cleardoublepage}{\cleardoublepageeven}
  \let\@makechapterhead\@gobble % make \@makechapterhead do nothing
  \chapter{#1}
  \addtocounter{chapter}{-1}
  \endgroup
}
\makeatother

% Formatting commands to use in text.
\newcommand{\figuretitle}[1]{{\bfseries #1}}
\newcommand{\latin}{\textit}
\newcommand{\chapterref}[2]{\mbox{\hyperref[#1]{\textbf{#2}}}}
\newcommand{\code}[1]{{\tt #1}}

% Bibliography
\usepackage[sectionbib]{natbib}
\usepackage{bibunits}
\defaultbibliographystyle{jxb_links}
\defaultbibliography{library}
\setlength{\bibsep}{1.2mm}
\addto{\captionsbritish}{\renewcommand{\bibname}{References}}

% Math and text commands.
\newcommand{\bmvec}[1]{\bm{\mathrm{#1}}}
\newcommand{\swimming}{\mathrm{Sw}}
\newcommand{\reynolds}{\mathrm{Re}}
\newcommand{\strouhal}{\mathrm{St}}
\newcommand{\CoT}{\mathrm{CoT}}
\renewcommand{\d}{\mathrm{d}}
\newcommand{\T}{\mathrm{T}}
\newcommand{\deriv}[2]{\frac{\d #1}{\d #2}}
\newcommand{\derivn}[3]{\frac{\d^#3 #1}{\d #2^#3}}
\newcommand{\pderiv}[2]{\frac{\partial #1}{\partial #2}}
\newcommand{\pderivn}[3]{\frac{\partial^#3 #1}{\partial #2^#3}}
\newcommand{\ci}{CI$_\textrm{95\%}$}
\newcommand{\scinum}[2]{#1~\cdot~10^{#2}}
\newcommand{\dimension}{\mathsf}



% Make sure paragraphs stay together - sometimes they are stretched apart automatically.
\setlength{\parskip}{2pt}

\begin{document}
\bibliographyunit[\chapter]


\chapter*{Supplemental information}
\renewcommand{\thesection}{S\arabic{section}}
\renewcommand{\thesubsection}{S\arabic{section}.\arabic{subsection}}
\renewcommand{\thesubsubsection}{S\arabic{section}.\arabic{subsection}.\arabic{subsubsection}}
\renewcommand{\theequation}{S\arabic{equation}}
\renewcommand{\thefigure}{S\arabic{figure}}
\renewcommand{\thetable}{S\arabic{table}}
\setcounter{section}{0}
\setcounter{subsection}{0}
\setcounter{equation}{0}
\setcounter{figure}{0}
\setcounter{table}{0}

\vspace*{3cm}
{\huge Bumblebees land rapidly by robustly \\ accelerating towards the surface during \\ visually guided landings}
\\*[0.5cm]
{\large Pulkit Goyal, Johan L. van Leeuwen, Florian T. Muijres}
\clearpage

\section{Figures}
\label{sec:s3_figures}
\begin{figure}[H]
	\centering
	\includegraphics{figures/chapter3/figure_s01_independence_with_f}
	\caption{\figuretitle{The effect of factor $\boldsymbol{f}$ on the results.} (A,B) The dependence of expansion-acceleration $\dot{r}_e$ (A) and mean acceleration $\overline{A}_e$ (B) on distance from the landing surface ($y_0$), step-change of relative rate of expansion required in an entry segment $\Delta r_e$, the set-point $r^*$, environmental light intensity and landing type (landing after a take-off or from a free-flight) for different factors $f$ (Equation~\ref{eq:ch03_bbee_rdot_amean_model}: $ \log(\dot{r}_e) \sim N(\alpha + \alpha_d + \alpha_a + \alpha_s + \beta_1~\log(y_{0~i,d,a,s}) + \beta_2~\textrm{MEDIUMlight}_{i,d,a,s} + \beta_3~\textrm{HIGHlight}_{i,d,a,s} + \beta_4~\textrm{fromTakeoff}_{i,d,a,s} + \beta_5~\log(\Delta r_{e~i,d,a,s}) + \beta_6~\log(r^*_{i,d,a,s}) + \beta_7~\log(y_{0~i,d,a,s})\times \log(\Delta r_{e~i,d,a,s}),~\sigma^2)$, similar equation holds for $\overline{A}_e$). The vertical bars for each coefficient indicate 95\% confidence intervals.}
	\label{figure_s01_independence_with_f}
\end{figure}

\begin{figure}[H]
	\centering
	\includegraphics{figures/chapter3/figure_s02_rdot_approximation}
	\caption{\figuretitle{During the transient phases of a landing approach, bumblebees keep the expansion acceleration $\bm{\dot{r}_e}$ approximately constant.} (A) Three examples of landing approaches in which the variation of relative rate of expansion $r$ during entry segments (blue) is fitted with time-to-touchdown $t$ using a linear regression (black). This linear regression approximates the optic expansion acceleration with a constant value. The constant-$r$ segment is shown in red and the black arrow indicates the variation of abscissa data as a bumblebee approaches the landing disc. (d) The goodness of fit of the linear regression model for all identified transient phases, in the six experimental conditions, as defined by the coefficient of determination ($R^2$). At each condition, we show a box plot and the coefficient of determination ($R^2$) at all transient phases (dots). The median coefficient of determination is above $0.98$ in all tested treatments.}
	\label{figure_s02_rdot_approximation}
\end{figure}

\begin{figure}[H]
	\centering
	\includegraphics{figures/chapter3/figure_s03_sysid_params}
	\caption{\figuretitle{The parameters identified using system identification.} (a) Gain $K$, (b) natural frequency $w$ and (c) damping ratio $D$. For each treatment, the data from different landing types (landings initiated from free-flight or take-off) are shown together. The blue box indicates interquartile range, and black lines indicate the maximum and minimum, respectively. The values that lie $1.5$ times the interquartile range away from the top or the bottom of the blue box are labelled as outliers (Low (L), medium (M) and high (H) light conditions, checkerboard (+) and spoke (x) landing patterns).}
	\label{figure_s03_sysid_params}
\end{figure}

\begin{figure}[H]
	\centering
	\includegraphics{figures/chapter3/figure_s04_amean}
	\caption{\figuretitle{The variation of instantaneous acceleration $\bm{A(t)}$ during entry segments with the set-point of relative rate of expansion $\bm{r^*}$ and landing type (landing from a free-flight or after take-off).} The data plotted here corresponds to 2,620 entry segments in which bumblebees exhibited positive mean acceleration ($\overline{A}_e > 0$).}
	\label{figure_s04_amean}
\end{figure}


\begin{figure}[H]
	\centering
	\includegraphics{figures/chapter3/figure_s05_rdot_freeflight}
	\caption{\figuretitle{The depiction of how bumblebees modulate their sensorimotor control response $\bm{\dot{r}_e}$ in different light conditions when they landed from free-flight.} (A) Effect of required step-change in relative rate of expansion in an entry segment $\Delta r_e$ and distance from the landing platform $y_0$ on $\dot{r}_e$, data points are shown for $r^* \in [2.28,3.28]~s^{-1}$, solid curves depict statistical model output, and are plotted for $r^*=2.78~s^{-1}$ (the median value). (B) Effect of set-point of relative rate of expansion $r^*$ on $\dot{r}_e$, data points are plotted for $\Delta r_e \in [1.28, 2.08]~s^{-1}$ and $y_0 \in [0.18, 0.24]~m$ (these intervals are centered around their median values). See Table~\ref{tb:ch03_rdot_modulation} for statistical model output. (A,B) Top, middle and bottom panels correspond to low, medium and high light conditions, respectively.}
	\label{figure_s05_rdot_freeflight}
\end{figure}

\begin{figure}[H]
	\centering
	\includegraphics{figures/chapter3/figure_s05_rdot_hastakeoff}
	\caption{\figuretitle{The depiction of how bumblebees modulate their sensorimotor control response $\bm{\dot{r}_e}$ in different light conditions when they landed after take-off.} (A) Effect of required step-change in relative rate of expansion in an entry segment $\Delta r_e$ and distance from the landing platform $y_0$ on $\dot{r}_e$, data points are shown for $r^* \in [2.28,3.28]~s^{-1}$, solid curves depict statistical model output, and are plotted for $r^*=2.78~s^{-1}$ (the median value). (B) Effect of set-point of relative rate of expansion $r^*$ on $\dot{r}_e$, data points are plotted for $\Delta r_e \in [1.28, 2.08]~s^{-1}$ and $y_0 \in [0.18, 0.24]~m$ (these intervals are centered around their median values). See Table~\ref{tb:ch03_rdot_modulation} for statistical model output. (A,B) Top, middle and bottom panels correspond to low, medium and high light conditions, respectively.}
	\label{figure_s05_rdot_hastakeoff}
\end{figure}

\begin{figure}[H]
	\centering
	\includegraphics{figures/chapter3/figure_s06_amean_freeflight}
	\caption{\figuretitle{The depiction of how bumblebees modulate their mean acceleration $\bm{\overline{A}_e}$ during an entry segment in different light conditions when they landed from free-flight.} (A) Effect of required step-change in relative rate of expansion in an entry segment $\Delta r_e$ and distance from the landing platform $y_0$ on $\overline{A}_e$, data points are shown for $r^* \in [2.28,3.28]~s^{-1}$, solid curves depict statistical model output, and are plotted for $r^*=2.78~s^{-1}$ (the median value). (B) Effect of set-point of relative rate of expansion $r^*$ on $\dot{r}_e$, data points are plotted for $\Delta r_e \in [1.28, 2.08]~s^{-1}$ and $y_0 \in [0.18, 0.24]~m$ (these intervals are centered around their median values). See Table~\ref{tb:ch03_amean_modulation} for statistical model output. (A,B) Top, middle and bottom panels correspond to low, medium and high light conditions, respectively.}
	\label{figure_s06_amean_freeflight}
\end{figure}


\begin{figure}[H]
	\centering
	\includegraphics{figures/chapter3/figure_s06_amean_hastakeoff}
	\caption{\figuretitle{The depiction of how bumblebees modulate their mean acceleration $\bm{\overline{A}_e}$ during an entry segment in different light conditions when they landed after take-off.} (A) Effect of required step-change in relative rate of expansion in an entry segment $\Delta r_e$ and distance from the landing platform $y_0$ on $\overline{A}_e$, data points are shown for $r^* \in [2.28,3.28]~s^{-1}$, solid curves depict statistical model output, and are plotted for $r^*=2.78~s^{-1}$ (the median value). (B) Effect of set-point of relative rate of expansion $r^*$ on $\dot{r}_e$, data points are plotted for $\Delta r_e \in [1.28, 2.08]~s^{-1}$ and $y_0 \in [0.18, 0.24]~m$ (these intervals are centered around their median values). See Table~\ref{tb:ch03_amean_modulation} for statistical model output. (A,B) Top, middle and bottom panels correspond to low, medium and high light conditions, respectively.}
	\label{figure_s06_amean_hastakeoff}
\end{figure}



\newpage
\section{Tables}
\label{sec:s3_tables}
%Table SX for variation in number of entry segments with factor f

\begin{table}[H]\centering
	\captionsetup{width=\textwidth, justification=justified}
	\caption{\figuretitle{Analysis of how bumblebees modulate the expansion-acceleration ($\bm{\dot{r}_e}$) during entry segments with the starting distance from the landing surface ($\bm{y_0}$), the required step-change in relative rate of expansion ($\bm{\Delta r_e}$), the final set-point to reach ($\bm{r^*}$), light conditions and the landing type.} The data comprises of 2,651 entry segments identified in 2,511 landing maneuvers of bumblebees (statistical model as given by Equation~\ref{eq:ch03_bbee_rdot_amean_model}: 
		$ \log(\dot{r}_{e~i,d,a,s}) \sim N(\alpha + \alpha_d + \alpha_a + \alpha_s + \beta_1~\log(y_{0~i,d,a,s}) + \beta_2~\textrm{MEDIUMlight}_{i,d,a,s} + \beta_3~\textrm{HIGHlight}_{i,d,a,s} + \beta_4~\textrm{fromTakeoff}_{i,d,a,s} + \beta_5~\log(\Delta r_{e~i,d,a,s}) + \beta_6~\log(r^*_{i,d,a,s}) + \beta_7~\log(y_{0~i,d,a,s})\times \log(\Delta r_{e~i,d,a,s}),~\sigma^2)$).}
	\label{tb:ch03_rdot_modulation}
	\begin{tabular}{p{3cm}CCCC}
		\toprule
		Fixed effect             & \text{Estimate} & \text{Std error} & \text{t value} & \text{Pr(\textgreater{}|t|)} \\
		\midrule
		
		$\alpha$ & 1.35  & 0.05 & 28.24  & 1.91E-10 \\
		$\beta_1$&-0.29 & 0.02 & -11.57 & 3.06E-30 \\
		$\beta_2$&0.03  & 0.02 & 2.07   & 0.039    \\
		$\beta_3$&0.10  & 0.01 & 6.55   & 7.08E-11 \\
		$\beta_4$&-0.02 & 0.01 & -1.99  & 0.047    \\
		$\beta_5$&-0.16 & 0.05 & -3.01  & 0.003    \\
		$\beta_6$&0.32  & 0.03 & 12.48  & 9.43E-35 \\
		$\beta_7$&-0.36 & 0.03 & -11.17 & 2.43E-28 \\	
		
		\bottomrule         
	\end{tabular}
\end{table}

\begin{table}[H]\centering
	\captionsetup{width=\textwidth, justification=justified}
	\caption{\figuretitle{Analysis of how the mean acceleration of bumblebees in an entry segment ($\bm{\overline{A}_e}$) varies with the starting distance from the landing surface ($\bm{y_0}$), the required step-change in relative rate of expansion ($\bm{\Delta r_e}$), the final set-point to reach ($\bm{r^*}$), light conditions and the landing type.} The data comprises of 2,620 entry segments identified in 2,485 landing maneuvers of bumblebees (statistical model as given by Equation~\ref{eq:ch03_bbee_rdot_amean_model}: 
		$ \log(\overline{A}_{e~i,d,a,s}) \sim N(\alpha + \alpha_d + \alpha_a + \alpha_s + \beta_1~\log(y_{0~i,d,a,s}) + \beta_2~\textrm{MEDIUMlight}_{i,d,a,s} + \beta_3~\textrm{HIGHlight}_{i,d,a,s} + \beta_4~\textrm{fromTakeoff}_{i,d,a,s} + \beta_5~\log(\Delta r_{e~i,d,a,s}) + \beta_6~\log(r^*_{i,d,a,s}) + \beta_7~\log(y_{0~i,d,a,s})\times \log(\Delta r_{e~i,d,a,s}),~\sigma^2)$).}
	\label{tb:ch03_amean_modulation}
	\begin{tabular}{p{3cm}CCCC}
		\toprule
		Fixed effect             & \text{Estimate} & \text{Std error} & \text{t value} & \text{Pr(\textgreater{}|t|)} \\
		\midrule
		$\alpha$ &1.77  & 0.08 & 21.42  & 3.12E-15 \\
		$\beta_1$&0.51  & 0.05 & 11.17  & 2.38E-28 \\
		$\beta_2$&0.04  & 0.03 & 1.54   & 0.123    \\
		$\beta_3$&0.14  & 0.03 & 5.28   & 1.38E-07 \\
		$\beta_4$&-0.11 & 0.02 & -5.12  & 3.37E-07 \\
		$\beta_5$&0.42  & 0.10 & 4.20   & 2.72E-05 \\
		$\beta_6$&-1.47 & 0.05 & -30.63 & 2.9E-176 \\
		$\beta_7$&-0.55 & 0.06 & -9.40  & 1.12E-20 \\
		
		
		\bottomrule         
	\end{tabular}
\end{table}

%
%\begin{table}\centering
%	\captionsetup{width=\textwidth, justification=justified}
%	\caption{\figuretitle{Analysis of how the landing performance of bumblebees, measured as the time ($\bm{\Delta t}$) they take to travel $\bm{0.2~m}$, varied with the environmental light conditions and the landing type (landing from take-off or free-flight).}. The data comprises of travel time of bumblebees to fly from $y = 0.25~m$ to $y = 0.05~m$ in 5,700 landing maneuvers. Post-hoc tests compare differences in travel time in the presence of different light conditions and landing types (statistical model as given by Equation~\ref{eq:ch03_travel_time_model}: $\Delta t \sim N(\alpha + \alpha_d + \alpha_s + \beta_1~\textrm{MEDIUMlight}_{i,d,s} + \beta_2~\textrm{HIGHlight}_{i,a,s} + \beta_3~\textrm{fromTakeoff}_{i,d,s},~\sigma^2)$).}
%	\label{tb:ch03_travel_time}
%	\begin{tabular}{p{3cm}CCCC}
%		\toprule
%		Fixed effect             & \text{Estimate} & \text{Std error} & \text{t value} & \text{Pr(\textgreater{}|t|)} \\
%		\midrule
%		$\alpha$ &0.42  & 0.01 & 42.73  & 7.04E-25 \\
%		$\beta_1$ &-0.03 & 0.01 & -3.05  & 0.002    \\
%		$\beta_2$ &-0.06 & 0.01 & -7.78  & 8.29E-15 \\
%		$\beta_3$ &-0.09 & 0.01 & -16.42 & 3.45E-59 \\
%		
%		\\
%		
%		Post-hoc constrasts* in $\Delta t$ & \text{Estimate} & \text{Std error} & \text{z ratio} & \text{$p$ value} \\
%		\midrule
%		L F - M F & 0.03 & 0.01 & 3.05  & 0.035    \\
%		L F - H F & 0.06 & 0.01 & 7.78  & 1.05E-13 \\
%		L F - L T & 0.09 & 0.01 & 16.42 & 2.19E-59 \\
%		L F - M T & 0.12 & 0.01 & 11.69 & 2.19E-30 \\
%		L F - H T & 0.15 & 0.01 & 15.88 & 1.4E-55  \\
%		M F - H F & 0.04 & 0.01 & 5.46  & 7.14E-07 \\
%		M F - L T & 0.07 & 0.01 & 7.00  & 3.84E-11 \\
%		M F - M T & 0.09 & 0.01 & 16.42 & 2.19E-59 \\
%		M F - H T & 0.13 & 0.01 & 15.16 & 9.95E-51 \\
%		H F - L T & 0.03 & 0.01 & 3.57  & 0.005    \\
%		H F - M T & 0.06 & 0.01 & 6.79  & 1.67E-10 \\
%		H F - H T & 0.09 & 0.01 & 16.42 & 2.19E-59 \\
%		L T - M T & 0.03 & 0.01 & 3.05  & 0.035    \\
%		L T - H T & 0.06 & 0.01 & 7.78  & 1.05E-13 \\
%		M T - H T & 0.04 & 0.01 & 5.46  & 7.14E-07 \\
%		
%		\\
%		\multicolumn{5}{l}{\multirow{1}{10cm}{*\scriptsize{Low (L), medium (M) and high (H) light conditions, free-flight (F) and take-off (T) starting conditions.}}} \\
%		\multicolumn{5}{l}{} \\
%		\bottomrule         
%	\end{tabular}
%\end{table}
%



\newpage
\section{Supporting text}
\label{sec:s3_text}

\subsection{Statistical models}
\label{sec:s3_text_stat_models}

We developed linear mixed-effects models to find how the transient response of the sensorimotor control system of landing bumblebees ($\dot{r}_e$) and the resulting mean accelerations ($\overline{A}_e$) varied with the starting distance from the landing surface ($y_0$), the required step-change in relative rate of expansion ($\Delta r_e$), the final set-point to reach ($r^*$), landing patterns (checkerboard and spoke), environmental light intensities, and the starting condition of the landing maneuver (whether the landing is from a free-flight or after a take-off). We first constructed a full model with aforementioned variables along with their interactions as fixed factors, and with the day of the experiment, the landing approach and the landing side (whether landing disc is located on the hive side or the food source side) as random intercepts. The model dredging revealed that the landing patterns did not affect either of the response variables ($\dot{r}_e$ or $\overline{A}_e$). Moreover, among all interaction terms, only $\text{log}(y_0)\times \text{log}(\Delta r_e)$ term was found to be significant, therefore we used the following reduced model:
\begin{equation}
	\label{eq:ch03_bbee_rdot_amean_model}
	\begin{array}{lll}
		\!\!\!\!\!\! \log(\dot{r}_{e~i,d,a,s}) \sim N( \!\!\!\!\!
		& \alpha + \alpha_d + \alpha_a + \alpha_s + \beta_1~\log(y_{0~i,d,a,s}) + \beta_2~\textrm{MEDIUMlight}_{i,d,a,s} + \\
		&\beta_3~\textrm{HIGHlight}_{i,d,a,s} + \beta_4~\textrm{fromTakeoff}_{i,d,a,s} + \beta_5~\log(\Delta r_{e~i,d,a,s}) + \\
		&\beta_6~\log(r^*_{i,d,a,s}) + \beta_7~\log(y_{0~i,d,a,s})\times \log(\Delta r_{e~i,d,a,s}),~\sigma^2)
	\end{array}                                          
\end{equation}
where $\dot{r}_{e~i,d,a,s}$, $y_{0~i,d,a,s}$, $\Delta r_{e~i,d,a,s}$, and $r^*_{i,d,a,s}$ are the measurements from the $i$-th entry segment from $d$-th day, $a$-th landing approach and $s$-th landing side, $\alpha$ is the regression intercept for the low light intensity and free-flight starting condition (overall intercept), $\alpha_d$ is the day-specific intercept, $\alpha_a$ is the landing-approach-specific intercept, $\alpha_s$ is the landing-side-specific intercept, $\textrm{MEDIUMlight}_{i,d,a,s}$, $\textrm{HIGHlight}_{i,d,a,s}$ and $\textrm{fromTakeoff}_{i,d,a,s}$ indicate if medium light condition, high light condition and take-off starting condition are present for the $i$-th measurement from $d$-th day, $a$-th landing approach and $s$-th landing side ($0$ = no, $1$ = yes), $\beta_i~\forall i \in \{2,3,4\}$ represent differences of fixed-effects from the overall intercept, $\beta_i~\forall i \in \{1,5,6,7\}$ represent the slopes of different covariates along with an interaction, and $\sigma$ is the residual standard deviation. The similar formula holds for the mean acceleration $\overline{A}_e$ as well. The statistical outputs are given in Tables~\ref{tb:ch03_rdot_modulation} and~\ref{tb:ch03_amean_modulation}.

%We also developed a statistical model to test how the landing performance of bumblebees, measured as the time ($\Delta t$) they take to travel $0.2~m$ (from $y = 0.25~m$ to $y = 0.05~m$), varied in different tested environmental conditions (light conditions and landing patterns) and the landing type (landing from take-off or free-flight). For this purpose, we first constructed a full model with light, pattern and landing type along with their interactions as fixed factors and with the day of the experiment and the landing side as random intercepts. We found that pattern did not affect the travel time. Therefore, we used the following reduced model:
%\begin{equation}
%	\label{eq:ch03_travel_time_model}
%	\begin{array}{lll}
%		\!\!\!\!\!\! \Delta t \sim N( \!\!\!\!\!
%		& \alpha + \alpha_d + \alpha_s + \beta_1~\textrm{MEDIUMlight}_{i,d,s} + 
%		\beta_2~\textrm{HIGHlight}_{i,a,s} + \\
%		& \beta_3~\textrm{fromTakeoff}_{i,d,s},~\sigma^2)
%	\end{array}                                          
%\end{equation}
%The statistical output for travel time in different tested conditions is given in Table~\ref{tb:ch03_travel_time}. 

\subsection{Governing equations for motion at a constant expansion-acceleration}
\label{sec:s3_text_const_rdot_motion}
During entry segments, the motion of an animal towards the landing surface can be well approximated by a motion at a constant expansion-acceleration ($\dot{r}$). Such a motion is described by a following system of equations:
\begin{ceqn}
	\begin{subequations}
		\label{eq:ch03_const_rdot_motion}
		\begin{align}
			V(t) & = - \frac{dy(t)}{dt} \\
			A(t) & =  \frac{dV(t)}{dt} = \dot{r}~y(t) - \frac{V^2(t)}{y(t)} \\
			y(t_0) & = y_0 \text{, and } V(t_0) = (r^*-\Delta r)~y_0
		\end{align}
	\end{subequations}
\end{ceqn}
where $y(t)$, $V(t)$, and $A(t)$ are the distance (in a direction normal to the landing surface), velocity and acceleration of an animal at time $t$, and $r^*$ and $\Delta r_e$ are the new set-point and the required step-change in relative rate of expansion at the moment of switching the set-point ($t_0$).



\end{document}